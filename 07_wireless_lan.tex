\section{Wireless LAN}

%7.1
\subsection{How is mobility restricted using WLANs? What additional elements are needed for
roaming between networks, how and where can WLANs support roaming? In your
answer, think of the capabilities of layer 2 where WLANs reside.}

%7.2
\subsection{What are the basic differences between wireless WANs and WLANs, and what are
the common features? Consider mode of operation, administration, frequencies,
capabilities of nodes, services, national/international regulations.}

%7.3
\subsection{With a focus on security, what are the problems of WLANs? What level of security
can WLANs provide, what is needed additionally and how far do the standards go?}

%7.4
\subsection{Compare IEEE 802.11, HiperLAN2, and Bluetooth with regard to their ad-hoc
capabilities. Where is the focus of these technologies?}

%7.5
\subsection{If Bluetooth is a commercial success, what are remaining reasons for the use of
infrared transmission for WLANs?}

%7.6
\subsection{Why is the PHY layer in IEEE 802.11 subdivided? What about HiperLAN2 and
Bluetooth?}

%7.7
\subsection{Compare the power saving mechanisms in all three LANs introduced in this chapter.
What are the negative effects of the power saving mechanisms, what are the tradeoffs
between power consumption and transmission QoS?}

%7.8
\subsection{Compare the offered QoS in all three LANs in ad hoc mode. What advantages does
an additional infrastructure offer? How is QoS provided in Bluetooth? Can one of the
LAN technologies offer hard QoS (i.e., not only statistical guarantees regarding a
QoS parameter)?}

%7.9
\subsection{How do IEEE 802.11, HiperLAN2 and Bluetooth, respectively, solve the hidden
terminal problem?}

%7.10
\subsection{How are fairness problems regarding channel access solved in IEEE 802.11,
HiperLAN2, and Bluetooth respectively? How is the waiting time of a packet ready to
transmit reflected?}

%7.11
\subsection{What different solutions do all three networks offer regarding an increased reliability
of data transfer?}

%7.12
\subsection{In what situations can collisions occur in all three networks? Distinguish between
collisions on PHY and MAC layer. How do the three wireless networks try to solve the
collisions or minimize the probability of collisions?}

%7.13
\subsection{Compare the overhead introduced by the three medium access schemes and the
resulting performance at zero load, light load, high load of the medium. How does the
number of collisions increase with the number of stations trying to access the
medium, and how do the three networks try to solve the problems? What is the
overall scalability of the schemes in number of nodes?}

%7.14
\subsection{How is roaming on layer 2 achieved, and how are changes in topology reflected?
What are the differences between infrastructure based and ad hoc networks
regarding roaming?}

%7.15
\subsection{What are advantages and problems of forwarding mechanisms in Bluetooth networks
regarding security, power saving, and network stability?}

%7.16
\subsection{Name reasons for the development of wireless ATM. What is one of the main
differences to Internet technologies from this point of view? Why did WATM not
succeed as stand-alone technology, what parts of WATM succeeded?}


