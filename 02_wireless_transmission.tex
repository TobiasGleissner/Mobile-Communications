\section{Wireless Transmission}

%2.1
\subsection{Frequency regulations may differ between countries. Check out the regulations valid
for your country (within Europe the European Radio Office may be able to help you,
www.ero.dk, for the US try the FCC, www.fcc.gov, for Japan ARIB, www.arib.or.jp).}

%2.2
\subsection{Why can waves with a very low frequency follow the earth’s surface? Why are they
not used for data transmission in computer networks?}
Nicht benutzt in Computernetzwerken weil man riesige Antennen benötigt um sie zu empfangen -> unpraktisch und teuer. Außerdem sind damit nur geringe Datenraten zu erreichen, da die Bandbreite gering ist.

%2.3
\subsection{Why does the ITU-R only regulate ‘lower’ frequencies (up to some hundred GHz) and
not higher frequencies (in the THz range)?}

%2.4
\subsection{What are the two different approaches in regulation regarding mobile phone systems
in Europe and the US? What are the consequences?}
In den USA Versteigerung der Frequenzen an den Meistbietenden. In Europa wird nach Technologie separiert. Konsequenz ist, dass in Amerika verschiedene Technologien auf dem gleichen Frequenzband mit einander konkurrieren und sich gegenseitig stören.

%2.5
\subsection{Why is the international availability of the same ISM bands important?}
Für den Export von Geräten wie Routern, Mikrowellen oder medizinischen Geräten ohne das die Funkmodule regional angepasst werden müssten? 

%2.6
\subsection{Is it possible to transmit a digital signal, e.g., coded as square wave as used inside a
computer, using radio transmission without any loss? Why?}
Man hat im Funk ja nur ein gewisses Frequenzband zur Verfügung. Ein digitales Signal besteht aber aus unendlich vielen analogen Signalen mit unterschiedlicher Frequenz. Es müssen also alle Frequenzen weggelassen werden, die nicht in der verfügbaren Bandbreite liegen -> Signal wird ungenauer.

%2.7
\subsection{Is a directional antenna useful for mobile phones? Why? How can the gain of an
antenna be improved?}

%2.8
\subsection{What are the main problems of signal propagation? Why do radio waves not always
follow a straight line? Why is reflection both useful and harmful?}

%2.9
\subsection{Name several methods for ISI mitigation. How does ISI depend on the carrier
frequency, symbol rate, and movement of sender/receiver? What are the influences
of ISI on TDM schemes?}

%2.10
\subsection{What are the means to mitigate narrowband interference? What is the complexity of
the different solutions?}

%2.11
\subsection{Why, typically, is digital modulation not enough for radio transmission? What are
general goals for digital modulation? What are typical schemes?}

%2.12
\subsection{Think of a phase diagram and the points representing bit patterns for a PSK scheme
(see Figure 2.29). How can a receiver decide which bit pattern was originally sent
when a received ‘point’ lies somewhere in between other points in the diagram? Why
is it, thus, difficult to code more and more bits per phase shift?}

%2.13
\subsection{What are the main benefits of a spread spectrum system? How can spreading be
achieved? What replaces the guard space in Figure 2.31 when compared to Figure
2.32? How can DSSS systems benefit from multipath propagation?}

%2.14
\subsection{What are the main reasons for using cellular systems? How is SDM typically realized
and combined with FDM? How does DCA influence the frequencies available in other
cells?}

%2.15
\subsection{What limits the number of simultaneous users in a TDM/FDM system compared to a
CDM system? What happens to the transmission quality of connections if the load
gets higher in a cell, i.e., how does an additional user influence the other users in the
cell?}



