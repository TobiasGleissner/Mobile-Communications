\section{Wireless Transmission}

%2.1
\subsection{Frequency regulations may differ between countries. Check out the regulations valid
for your country (within Europe the European Radio Office may be able to help you,
www.ero.dk, for the US try the FCC, www.fcc.gov, for Japan ARIB, www.arib.or.jp).}

%2.2
\subsection{Why can waves with a very low frequency follow the earth’s surface? Why are they
not used for data transmission in computer networks?}
Nicht benutzt in Computernetzwerken weil man riesige Antennen benötigt um sie zu empfangen $\rightarrow$ unpraktisch und teuer. Außerdem sind damit nur geringe Datenraten zu erreichen, da die Bandbreite gering ist.

%2.3
\subsection{Why does the ITU-R only regulate ‘lower’ frequencies (up to some hundred GHz) and
not higher frequencies (in the THz range)?}

%2.4
\subsection{What are the two different approaches in regulation regarding mobile phone systems
in Europe and the US? What are the consequences?}
In den USA Versteigerung der Frequenzen an den Meistbietenden. In Europa wird nach Technologie separiert. Konsequenz ist, dass in Amerika verschiedene Technologien auf dem gleichen Frequenzband mit einander konkurrieren und sich gegenseitig stören.

%2.5
\subsection{Why is the international availability of the same ISM bands important?}
Für den Export von Geräten wie Routern, Mikrowellen oder medizinischen Geräten ohne das die Funkmodule regional angepasst werden müssten.

%2.6
\subsection{Is it possible to transmit a digital signal, e.g., coded as square wave as used inside a
computer, using radio transmission without any loss? Why?}
Man hat im Funk ja nur ein gewisses Frequenzband zur Verfügung. Ein digitales Signal besteht aber aus unendlich vielen analogen Signalen mit unterschiedlicher Frequenz nach der Fouriertransformation. Es müssen also alle Frequenzen weggelassen werden, die nicht in der verfügbaren Bandbreite liegen $\rightarrow$ Signal wird ungenauer.

%2.7
\subsection{Is a directional antenna useful for mobile phones? Why? How can the gain of an
antenna be improved?} 
Nein, also nicht direkt am Handy bzw. am UE/MS. Man weiß ja nicht in welcher Richtung sich die gerade verbundene Base Station befindet, bzw. man will auch nicht gezwungen sein, das Handy in einer ortsabhängigen Position zu tragen. Es sollte also in alle Richtungen gleichmäßig gesendet werden. Zumal die Zelle ja auch regelmäßig durch die implizierte Mobilität gewechselt wird.

Für Basisstationen, die nur bestimmte Bereiche abecken sollen ist es widerrum sinnvoll, weil weniger Energie benötigt wird.

Gain Improvement: Benutze MIMO, mehr kleine Antennen $\rightarrow$ weniger Energie wird benötigt

%2.8
\subsection{What are the main problems of signal propagation? Why do radio waves not always
follow a straight line? Why is reflection both useful and harmful?}
\begin{itemize}

\item Fading: Frequenzabhängig, Umwelt beeinflusst das Signal indem es die Wellenlänge vergrößert oder verringert

\item Shadowing: Abschirmung bestimmter Frequenzen durch Materialien wie Stahlbetonwände, Bäume, Wasser. Je höher die Frequenz, desto stärker treten shadowing-Effekte auf.

\item Reflection: Reflexion von Signalen an Wänden, etc. Besonders stark in Innenstädten mit engen Gassen

\item Refraction: Wie beim Licht im Wasser. Ablenkung des Signals durch ein Medium

\item Scatter: Brechung des Signals wie beim Licht und einem Prisma. Spaltet Signal in seine Frequenzbestandteile auf.

\item Diffraction an Kanten. Ablenkung + Aufspaltung des Signals.

\end{itemize}

Frage 2 ergibt sich aus obigen Problemen. Signale haben selten freie Bahn, sondern kollidieren mit Objekten und werden abgelenkt.
Reflexion hilft dabei schlecht erreichbare Ecken in z.B. Innenstädten abzudecken, ohne zusätzliche Antennen aufstellen zu müssen. Allerdings lenkt sie auch das Signal ab, was dazu führt, das gesendete Signale nicht dort ankommen wo sie hin sollen oder zu spät und in schlechter Qualität am Empfänger eintreffen.

%2.9
\subsection{Name several methods for ISI mitigation. How does ISI depend on the carrier
frequency, symbol rate, and movement of sender/receiver? What are the influences
of ISI on TDM schemes?}
ISI means intersymbol interference

\begin{itemize}

\item Kürzere Impulse nutzen, sodass möglichst wenig Energie sich auf die benachbarten Symbole auswirken kann

\item Zeitliche Trennung von Symbolen durch guard-periods. 

\item Equalizer im Receiver, der versucht die Effekte rückgängig zu machen

\item Sequence Detector im Receiver um mithilfe des Viterbi Algorithmus die übertragene Symbolfrequenz zu schätzen.

\end{itemize}

Carrier Frequenz hoch $\rightarrow$ stärkere Interferenzen, weil die Impulse näher beieinander liegen

Symbol Rate hoch $\rightarrow$ mehr Symbole pro Impuls $\rightarrow$ \"schöneres\" Signal notwendig zum decodieren $\rightarrow$ mehr Interferenzen und kaputte Signale

Bewegungen von Sender und Empfänger führen zu unterschiedlichen Übertragungszeiten der einzelnen Signale. Gerade wenn diese Bewegungen sehr schnell sind, treffen Signale eventuell zu einem unerwarteten Zeitpunkt ein (auch zugleich) und stören sich.

ISI bedeutet auch, dass eventuell die Signale unterschiedlicher Time Slots miteinander interferieren und somit Signale an einer Stelle erscheinen, wo sie nichts zu suchen haben. Behoben durch guard spaces zwischen den Time Slots und früheres Senden der Nachrichten (Time Advance bei GSM).


%2.10
\subsection{What are the means to mitigate narrowband interference? What is the complexity of
the different solutions?}
Spread Spectrum.

Direct Sequence Spread Spectrum $\rightarrow$ XOR von Signal(-bit) mit einer Zufallszahl (der sog. Chipping-Sequence). Signale, die orthogonale Chipping-Sequences verwenden, können voneinander getrennt werden. Je mehr Chips pro Bit, desto breiter das Signal. \todo{breiter=mehr Spektrum, stimmt das mit dem breiteren Signal wirklich?}

Frequency Hopping Spread Spectrum $\rightarrow$ Sprunghaftes ändern der Trägerfrequenz (Bluetooth)
Fast Hopping: mehrere Frequenzen per Bit, Slow Hopping: mehrere Bits per Frequenz

FHSS ist deutlich einfacher zu implementieren, da einfach in einem gewissen Takt gesprungen werden muss. Bei DSSS muss hingegen bei Sender und Empfänger das Signal umgerechnet werden was Prozessorleistung erfordert.

%2.11
\subsection{Why, typically, is digital modulation not enough for radio transmission? What are
general goals for digital modulation? What are typical schemes?}
Sender unterschiedlicher Dienste (z.B. Zellen, Radiosender) haben bestimmte zugewiesene  
Frequenzbänder in denen sie sich bewegen können ohne andere Sendestationen zu stören. Deshalb muss das Signal mit den Nutzdaten noch auf die genutzte Trägerfrequenz moduliert werden.

Ziele von digitaler Modulation sind spektrale Effizienz (so viel Information wie möglich pro Frequenzintervall), Stromsparen und robuste Signale.

Digitale Modulation nennt sich auch Shift Keying
\begin{itemize}

\item Amplitude Shift Keying: Verändere Ausschlag der Welle passend zum zu übertragenden Bit

\item Frequency Shift Keying: Ändere Frequenz passend zu 0 oder 1

\item Phase Shift Keying: Verschiebe die Phase des Signals abhängig vom Bit-wert.

\end{itemize}

%2.12
\subsection{Think of a phase diagram and the points representing bit patterns for a PSK scheme
(see Figure 2.29). How can a receiver decide which bit pattern was originally sent
when a received ‘point’ lies somewhere in between other points in the diagram? Why
is it, thus, difficult to code more and more bits per phase shift?}
Wir nehmen den Punkt, der am nächsten liegt? Wenn man sich des Quadranten sicher ist, weiß man zumindestens die ersten beiden bits.

Wenn man viele Bits per Phase verwendet, resultiert das in mehr Punkten in unserer Map. Dadurch sind die Bereiche in denen ein Bit eindeutig einem Wert zugeordnet werden kann kleiner und das Signal ist weniger robust.


%2.13
\subsection{What are the main benefits of a spread spectrum system? How can spreading be
achieved? What replaces the guard space in Figure 2.31 when compared to Figure
2.32? How can DSSS systems benefit from multipath propagation?}
\underline{Benefits:}
\begin{itemize}
\item Störungen, die nur einen kleinen Frequenzbereich betreffen stellen kein Problem mehr dar, da das Signal sich über einen größeren Bereich erstreckt. Das Signal wird also robuster.
\item Mit einer geheimen und wechselnden Chipping Sequence eine Verschlüsselung zu erzielen.
\item Nutzer können auch auf dem selben Frequenzband gleichzeitig senden, wenn sie unterschiedliche Codes zum spreaden nutzen.
\item Es muss keine Frequenzplanung im Bezug auf die Position der Mobilfunkmasten mehr durchgeführt werden
\end{itemize}

\underline{Spreading:} Signal $\rightarrow$ xor mit Chipping-Sequence $\rightarrow$ digitale modulation und modulation auf Trägerfrequenz

%guard space auf folie 2.34
\underline{Guard spaces:} Wird durch orthogonale Chipping-Sequences ersetzt.

\underline{Benefits from Multipathpropagation:} Offensichtlich nutzen DSSS Systeme sog. rake receivers (\todo{wie funktioniert das?}). Dadurch können sie leicht verschiedene Signale einfangen und zum Ausgangssignal rekombinieren.


%2.14
\subsection{What are the main reasons for using cellular systems? How is SDM typically realized
and combined with FDM? How does DCA influence the frequencies available in other
cells?}
\underline{Reasons cellular:} Mehr Nutzer möglich, weniger Energie für die Übertragung notwendig wegen kürzerer Wege, dezentral daher robuster.

\underline{Realization SDM+FDM:} Jede Zelle deckt einen bestimmten Bereich ab. Alle Nutzer in diesem Bereich nutzen die Basisstation dieser Zelle(SDM).
Jede Zelle hat ein bestimmtes Frequenzband, welches disjunkt zum Frequenzband benachbarter Zellen ist(FDM). 

DCA means dynamic channel allocation
\underline{DCA:} Wenn eine Zelle viele Nutzer hat, \"leiht\" sie von benachbarten Zellen dynamisch unbenutzte Frequenzen.

%2.15
\subsection{What limits the number of simultaneous users in a TDM/FDM system compared to a
CDM system? What happens to the transmission quality of connections if the load
gets higher in a cell, i.e., how does an additional user influence the other users in the
cell?}
\underline{user limitation:} TDM/FDM ist irgendwann voll belegt und lehnt dann einfach neue Nutzer ab, weil die verfügbaren Frequenzen bzw. die Timeslots aufgebraucht sind.

\underline{higher load in CDM:} CDM Je mehr Nutzer/Load in einer Zelle mit CDM, desto stärker das Rauschen $\rightarrow$ geringere Reichweite der Basisstation $\rightarrow$ irgendwann fallen entferntere Nutzer aus der Zelle.



