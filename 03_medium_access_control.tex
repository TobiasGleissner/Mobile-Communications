\section{Medium Access Control}

%3.1 
\subsection{What is the main physical reason for the failure of many MAC schemes known from
wired networks? What is done in wired networks to avoid this effect?}
Signalstärke -> hidden Terminals


%3.2 
\subsection{Recall the problem of hidden and exposed terminals. What happens in the case of
such terminals if Aloha, slotted Aloha, reservation Aloha, or MACA is used?}
\begin{itemize}

\item Aloha: Hidden: Terminals A und C senden einfach fröhlich drauflos -> Kollision -> Kein ACK von B -> neue Übertragung bis die Daten irgendwann ankommen

Exposed: Es wird einfach gesendet und damit auch keine Rücksicht auf irgendwelche belegten Medien genommen

\item slotted Aloha: s.o. Vielleicht Probleme beim Synchronisieren der Time Slots? Wobei das dann vermutlich einfach über das Terminal B laufen würde.

\item reservation Aloha: Synchronisation und Reservierung über Terminal B? 

\item MACA: Hidden: A sendet RTS -> B sendet CTS -> C empfängt CTS von B und wartet deshalb

Exposed: B möchte etwas zu A senden, C irgendwo anders hin (zu D). C empfängt kein CTS von A, da A nicht sichtbar und muss deshalb nicht warten

\end{itemize}

%3.3 
\subsection{How does the near/far effect influence TDMA systems? What happens in CDMA
systems? What are countermeasures in TDMA systems, what about CDMA systems?}
TDMA gar nicht? Also ich wüsste nicht weshalb. Jedes Terminal hat seinen Zeitslot und sendet da auf seiner Frequenz.

Bei CDMA kann ein weiter entferntes Terminal von einem Näheren übertönt werden. Eine Gegenmaßnahme wäre die Signalstärke anzupassen, wenn keine ACKs vom Receiver kommen bzw. vielleicht den näheren Terminals zu signalisieren schwächer zu senden. Also entfernungsabhängiges power-management.


%3.4 
\subsection{Who performs the MAC algorithm for SDMA? What could be possible roles of mobile
stations, base stations, and planning from the network provider?}

Der Provider legt die Zellstrukturen fest, also die Netztopologie. Also ist der Provider derjenige, der für den SDMA zuständig ist.

Die MS suchen sich über die Signalstärke die nächstgelegene BS und tauschen sich mit dieser aus. 

%3.5 
\subsection{What is the basic prerequisite for applying FDMA? How does this factor increase
complexity compared to TDMA systems? How is MAC distributed if we consider the
whole frequency space as presented in chapter 1?}


%3.6 
\subsection{Considering duplex channels, what are alternatives for implementation in wireless
networks? What about typical wired networks?}
FDD, also eine Menge an Frequenzen für den Uplink und eine Menge für den Downlink oder TDD als in einer bestimmten Zeit wird gesendet und in der Restlichen empfangen.

%3.7
\subsection{What are the advantages of a fixed TDM pattern compared to random, demand
driven TDM? Compare the efficiency in the case of several connections with fixed
data rates or in the case of varying data rates. Now explain why traditional mobile
phone systems use fixed patterns, while computer networks generally use random
patterns. In the future, the main data being transmitted will be computer-generated
data. How will this fact change mobile phone systems?}

%3.8 
\subsection{Explain the term interference in the space, time, frequency, and code domain. What
are countermeasures in SDMA, TDMA, FDMA, and CDMA systems?}

%3.9 
\subsection{Assume all stations can hear all other stations. One station wants to transmit and
senses the carrier idle. Why can a collision still occur after the start of transmission?}

%3.10 
\subsection{What are benefits of reservation schemes? How are collisions avoided during data
transmission, why is the probability of collisions lower compared to classical Aloha?
What are disadvantages of reservation schemes?}

%3.11 
\subsection{How can MACA still fail in case of hidden/exposed terminals? Think of mobile stations
and changing transmission characteristics.}

%3.12 
\subsection{Which of the MAC schemes can give hard guarantees related to bandwidth and
access delay?}

%3.13 
\subsection{How are guard spaces realised between users in CDMA?}

%3.14 
\subsection{Redo the simple CDMA example of subsection 3.5, but now add random ‘noise’ to the
transmitted signal $(–2,0,0,–2,+2,0)$. Add, for example, $(1,–1,0,1,0,–1)$. In this case,
what can the receiver detect for sender A and B respectively? Now include the
near/far problem. How does this complicate the situation? What would be possible
countermeasures?}