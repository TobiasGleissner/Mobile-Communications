\section{Mobile transport layer}

%9.1
\subsection{Compare the different types of transmission errors that can occur in wireless and
wired networks. What additional role does mobility play?}

%9.2
\subsection{What is the reaction of standard TCP in case of packet loss? In what situation does
this reaction make sense and why is it quite often problematic in the case of wireless
networks and mobility?}

%9.3
\subsection{Can the problems using TCP be solved by replacing TCP with UDP? Where could
this be useful and why is it quite often dangerous for network stability?}

%9.4
\subsection{How and why does I-TCP isolate problems on the wireless link? What are the main
drawbacks of this solution?}

%9.5
\subsection{Show the interaction of mobile IP with standard TCP. Draw the packet flow from a
fixed host to a mobile host via a foreign agent. Then a handover takes place. What
are the following actions of mobile IP and how does TCP react?}

%9.6
\subsection{Now show the required steps during handover for a solution with a PEP. What are the
state and function of foreign agents, home agents, correspondent host, mobile host,
PEP and care-of address before, during, and after handover? What information has
to be transferred to which entity in order to maintain consistency for the TCP
connection?}

%9.7
\subsection{What are the influences of encryption on the proposed schemes? Consider for
example IP security that can encrypt the payload, i.e., the TCP packet.}

%9.8
\subsection{Name further optimisations of TCP regarding the protocol overhead which are
important especially for narrow band connections. Which problems may occur?}

%9.9
\subsection{Assume a fixed Internet connection with a round trip time of 20 ms and an error rate
of 10-10. Calculate the upper bound on TCP’s bandwidth for a maximum segment size
of 1000 byte. Now two different wireless access networks are added. A WLAN with 2
ms additional one-way delay and an error rate of 10-3, and a GPRS network with an
additional RTT of 2 s and an error rate of 10-7. Redo the calculation ignoring the fixed
network’s error rate. Compare these results with the ones derived from the second
formula (use RTO = 5 RTT). Why are some results not realistic?}

%9.10
\subsection{Why does the link speed not appear in the formulas presented to estimate TCP’s
throughput? What is wrong if the estimated bandwidth is higher than the link speed?}




