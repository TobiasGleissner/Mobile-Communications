\section{Telecommunication systems}

%4.1
\subsection{Name some key features of the GSM, DECT, TETRA, and UMTS systems. Which
features do the systems have in common? Why have the older three different
systems been specified? In what scenarios could one system replace another? What
are the specific advantages of each system?}
\begin{itemize}

\item GSM (Global System for Mobilecommunication): Circuit Switched, wireless Communiaction, wide area coverage, 9,6 kbit/s bis 50kbit/s, voice, MMS, SMS

\item DECT (Digital Enhanced Cordless Telecommunications): Reichweite 30-50 m

\item TETRA: 

\item UMTS:

\end{itemize}

%4.2
\subsection{What are the main problems when transmitting data using wireless systems that were
made for voice transmission? What are the possible steps to mitigate the problems
and to raise efficiency? How can this be supported by billing?}

%4.3
\subsection{Which types of different services does GSM offer? Name some examples and give
reasons why these services have been separated.}
\begin{itemize}
\item Bearer Services are Services to transfer data between access points. OSI layers 1-3

\item Tele Services that enable voice communiaction via mobile phones (mobile telephony, emergency numbers, multinumbering, SMS, mailbox, fax, electronic mail)

\item Supplementary services similar to ISDN services (forward caller number, suppression of caller number, automatic call-back, conferencing, blocking incoming and outgoing calls)
\end{itemize}


%4.4
\subsection{Compared to the TCHs offered, standard GSM could provide a much higher data rate
(33.8 kbit/s) when looking at the air interface. What lowers the data rates available to
a user?}
Kontrollsignale, Guard Spaces zwischen den TDMA Slots, Error Correction also Redundanz in den Nutzerdaten

%4.5
\subsection{Name the main elements of the GSM system architecture and describe their
functions. What are the advantages of specifying not only the radio interface but also
all internal interfaces of the GSM system?}
\begin{itemize}
\item Mobile Station ist das Terminal mit dem der Nutzer auf das GSM Netz zugreifen kann

\item Base Station (BS Controller und B Transciever Station) sind die Zugriffspunkte des GSM Netzes zu denen sich die MS verbindet. Sie ist verantwortlich für Handover, Senden und Empfangen von Daten und dem zuweisen eines Channels und Time Slots.

\item Mobile Services Switching Center steuert alle Verbindungen auf zugehörigen BSCs

\item Datenbanken (HLR, VLR) enthalten Daten aller Nutzer eines Providers/einer Gebietes und enthalten u.a. den letzten Aufenthaltsort, die Nummer, Vertragsinformationen

\item Authentication Center für Identifizierung gegenüber dem Netz, Verschlüsselung

\item Equipment Identity Register 

\item Operation and Maintenance Center 
\end{itemize}


%4.6
\subsection{Describe the functions of the MS and SIM. Why does GSM separate the MS and
SIM? How and where is user-related data represented/stored in the GSM system?
How is user data protected from unauthorised access, especially over the air
interface? How could the position of an MS (not only the current BTS) be localised?
Think of the MS reports regarding signal quality.}



%4.7
\subsection{Looking at the HLR/VLR database approach used in GSM—how does this
architecture limit the scalability in terms of users, especially moving users?}

%4.8
\subsection{Why is a new infrastructure needed for GPRS, but not for HSCSD? Which
components are new and what is their purpose?}

%4.9
\subsection{What are the limitations of a GSM cell in terms of diameter and capacity (voice, data)
for the traditional GSM, HSCSD, GPRS? How can the capacity be increased?}

%4.10
\subsection{What multiplexing schemes are used in GSM for what purposes? Think also of other
layers apart from the physical layer.}

%4.11
\subsection{How is synchronisation achieved in GSM? Who is responsible for synchronisation
and why is synchronisation very important?}

%4.12
\subsection{What are the reasons for the delays in a GSM system for packet data traffic?
Distinguish between circuit-switched and packet-oriented transmission.}

%4.13
\subsection{Where and when can collisions occur while accessing the GSM system? Compare
possible collisions caused by data transmission in standard GSM, HSCSD, and
GPRS.}

%4.14
\subsection{Why and when are different signalling channels needed? What are the differences?}

%4.15
\subsection{How is localisation, location update, roaming, etc. done in GSM and reflected in the
data bases? What are typical roaming scenarios?}

%4.16
\subsection{Why are so many different identifiers/addresses (e.g., MSISDN, TMSI, IMSI) needed
in GSM? Give reasons and distinguish between user related and system related
identifiers.}

%4.17
\subsection{Give reasons for a handover in GSM and the problems associated with it. Which are
the typical steps for handover, what types of handover can occur? Which resources
need to be allocated during handover for data transmission using HSCSD or GPRS
respectively? What about QoS guarantees?}

%4.18
\subsection{What are the functions of authentication and encryption in GSM? How is system
security maintained?}

%4.19
\subsection{How can higher data rates be achieved in standard GSM, how is this possible with
the additional schemes HSCSD, GPRS, EDGE? What are the main differences of the
approaches, also in terms of complexity? What problems remain even if the data rate
is increased?}

%4.20
\subsection{What limits the data rates that can be achieved with GPRS and HSCSD using real
devices (compared to the theoretical limit in a GSM system)?}

%4.21
\subsection{Using the best delay class in GPRS and a data rate of 115.2 kbit/s – how many bytes
are in transit before a first acknowledgement from the receiver could reach the sender
(neglect further delays in the fixed network and receiver system)? Now think of typical
web transfer with 10 kbyte average transmission size—how would a standard TCP
behave on top of GPRS (see chapter 9 and chapter 10)? Think of congestion
avoidance and its relation to the round-trip time. What changes are needed?}

%4.22
\subsection{How much of the original GSM network does GPRS need? Which elements of the
network perform the data transfer?}

%4.23
\subsection{What are typical data rates in DECT? How are they achieved considering the TDMA
frames? What multiplexing schemes are applied in DECT for what purposes?
Compare the complexity of DECT with that of GSM.}

%4.24
\subsection{Who would be the typical users of a trunked radio system? What makes trunked radio
systems particularly attractive for these user groups? What are the main differences
to existing systems for that purpose? Why are trunked radio systems cheaper
compared to, e.g., GSM systems for their main purposes?}

%4.25
\subsection{Summarise the main features of 3rd generation mobile phone systems. How do they
achieve higher capacities and higher data rates? How does UMTS implement
asymmetrical communication and different data rates?}

%4.26
\subsection{Compare the current situation of mobile phone networks in Europe, Japan, China,
and North America. What are the main differences, what are efforts to find a common
system or at least interoperable systems?}

%4.27
\subsection{What disadvantage does OVSF have with respect to flexible data rates? How does
UMTS offer different data rates (distinguish between FDD and TDD mode)?}

%4.28
\subsection{How are different DPDCHs from different UEs within one cell distinguished in UTRA
FDD?}

%4.29
\subsection{Which components can perform combining/splitting at what handover situation? What
is the role of the interface Iur? Why can CDMA systems offer soft handover?}

%4.30
\subsection{How does UTRA-FDD counteract the near-far effect? Why is this no problem in
GSM?}


