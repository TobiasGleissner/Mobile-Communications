\section{Mobile network layer}

%8.1
\subsection{Recall routing in fixed IP networks (Kurose, 2003). Name the consequences and
problems of using IP together with the standard routing protocols for mobile
communications.}

%8.2
\subsection{What could be quick ‘solutions’ and why do they not work?}

%8.3
\subsection{Name the requirements for a mobile IP and justify them. Does mobile IP fulfil them
all?}

%8.4
\subsection{List the entities of mobile IP and describe data transfer from a mobile node to a fixed
node and vice versa. Why and where is encapsulation needed?}

%8.5
\subsection{How does registration on layer 3 of a mobile node work?}

%8.6
\subsection{Show the steps required for a handover from one foreign agent to another foreign
agent including layer 2 and layer 3.}

%8.7
\subsection{Explain packet flow if two mobile nodes communicate and both are in foreign
networks. What additional routes do packets take if reverse tunnelling is required?}

%8.8
\subsection{Explain how tunnelling works in general and especially for mobile IP using IP-in-IP,
minimal, and generic routing encapsulation, respectively. Discuss the advantages
and disadvantages of these three methods.}

%8.9
\subsection{Name the inefficiencies of mobile IP regarding data forwarding from a correspondent
node to a mobile node. What are optimizations and what additional problems do they
cause?}

%8.10
\subsection{What advantages does the use of IPv6 offer for mobility? Where are the entities of
mobile IP now?}

%8.11
\subsection{What are general problems of mobile IP regarding security and support of quality of
service?}

%8.12
\subsection{What is the basic purpose of DHCP? Name the entities of DHCP.}

%8.13
\subsection{How can DHCP be used for mobility and support of mobile IP?}

%8.14
\subsection{Name the main differences between multi-hop ad hoc networks and other networks.
What advantages do these ad hoc networks offer?}

%8.15
\subsection{Why is routing in multi-hop ad hoc networks complicated, what are the special
challenges?}

%8.16
\subsection{Recall the distance vector and link state routing algorithms for fixed networks. Why
are both difficult to use in multi-hop ad hoc networks?}

%8.17
\subsection{What are the differences between AODV and the standard distance vector algorithm?
Why are extensions needed?}

%8.18
\subsection{How does dynamic source routing handle routing? What is the motivation behind
dynamic source routing compared to other routing algorithms from fixed networks?}

%8.19
\subsection{How does the symmetry of wireless links influence the routing algorithms proposed?}

%8.20
\subsection{Why are special protocols for the support of micro mobility on the network layer
needed?}

%8.21
\subsection{What are the benefits of location information for routing in ad hoc networks, which
problems do arise?}

%8.22
\subsection{Think of ad hoc networks with fast moving nodes, e.g., cars in a city. What problems
arise even for the routing algorithms adapted to ad hoc networks? What is the
situation on highways?}






